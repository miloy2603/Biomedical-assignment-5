\documentclass[12pt,a4paper]{report}
\usepackage[utf8]{inputenc}
\usepackage{graphicx}



\begin{document} 


\begin{center}
    \Large{\sffamily{\textbf{NATIONAL INSTITUTE OF TECHNOLOGY}}}\\
\end{center}

\begin{center}
    \Large{\sffamily{RAIPUR , CHHATTISGARH}}\\ 
\end{center}

\begin{figure}
    \centering
    \includegraphics[scale=0.5]{nitlogo.png}
\end{figure}

\begin{center}
   \huge{\texttt{Assignment-5\\ on\\ "EMERGING TECHNOLOGIES IN HEALTHCARE"}}
  \end{center}
 
  
\begin{center}
\textbf{\underline{SUBMITTED BY:-}}\\

NAME:- MILOY KUMAR MANDAL\\
ROLL NO:- 21111032\\
BRANCH:- BIOMEDICAL ENGINEERING\\
SECTION:- A\\
SEMESTER:- 1ST\\
YEAR:- 1ST\\

\textbf{\underline{SUBMITTED TO:-}}\\
MR.SAURABH GUPTA\\
BIOMEDICAL DEPARTMENT\\
NIT RAIPUR\\


 
\end{center} 
\clearpage

\begin{center}
  \huge{\textbf{EMERGING TECHNOLOGIES IN HEALTHCARE}}
\end{center}

Coronavirus has become an uncommon disturbance to all aspects of the medical care industry in a short amount of time. In spite of the fact that the healthcare technology industry has been slow growing previously, development was needed to manage the pandemic. Artificial intelligence in medical care, as well as other significant advances, are critical to settling the crisis and for creating future development.\par

The technological advancement in the healthcare industry is molding the world to a better future. Advancements, for example, AR/VR, Artificial Intelligence, Robotics, 3D printing, and Nanotechnology are reforming the operations of healthcare companies . 2020 has been challenging for the healthcare industry as an industry with more roles and obligations in the pandemic.\par

2021 and the situation is as yet unsure regarding the pandemic, however,  technological progressions will never stop to facilitate the process and have a better outcome in the healthcare industry.Some of the emerging technologies in healthcare are:- \par

\begin{center}
  \small{\textbf{The Internet of Medical Things (IoMT)}}
\end{center}

Different devices and mobile apps have come to play an important part in tracking and forestalling persistent ailments for some patients and their doctors. By consolidating IoT development with telemedicine and telehealth innovations, a new Internet of Medical Things (IoMT) has arisen. This methodology incorporates the utilization of various wearables, including ECG and EKG monitors. Numerous other medical estimations can likewise be taken, for example, skin temperature, glucose level, and pulse readings.By 2025, the IoT business will be worth 6.2 trillion US dollars. The healthcare services industry has gotten so dependent on IoT innovation in 2020 that 30percent of that market share for IoT devices will come from healthcare.\par

\begin{center}
  \small{\textbf{Telemedicine}}
\end{center}

Telemedicine, or the act of clinicians seeing patients virtually instead of in physical workplaces and medical clinics, has expanded colossally during the pandemic as populations around the globe have restricted physical experiences. This practice has shown that remote consultations are conceivable, yet in addition easy and often preferable. A few experts state that this is just the start, and soon the size of telemedicine will increase.\par


\begin{center}
  \small{\textbf{Virtual Reality}}
\end{center}

The patient and doctor’s lives are overhauled with virtual reality. Later on, while you get worked on and to divert the pain patients taken to a vacation location. The development and the effect are failing to meet expectations for virtual reality innovation starting in 2020, yet the coming years will consistently be productive. The technologies are helpful for patients in pain management. Additionally, women are furnished with virtual reality headset to forget the labor pain.\par



\begin{center}
  \small{\textbf{3D-Printing}}
\end{center}

 
The future go for medical services is comprehensively observed with 3D printing technology with printing the body tissue till the artificial appendages, veins, pills, and some more. Organizations are delivering skin tissues with the platelets that help in supplanting the skin burn, and other skin related issues faced by the patients. The medications printed from 3D printing advances have been used since 2015 that are endorsed by the FDA.\par

 \begin{center}
  \small{\textbf{Nanotechnology}}
\end{center}

 The future in the healthcare industry is about nanotechnology, the headway in nanotechnology will help streamline the treatments. Organizations are giving a nano-pill camera that is utilized to study inside part of the body and assists with treating patients better. The coming years will assist us with better nanotechnology pills, nano-particles will go about as the drug delivery system, especially in treating cancer.\par

 \begin{center}
  \small{\textbf{Cloud Computing}}
\end{center}

Data collection and record keeping are an indispensable part of medical services, and truly, the management of this data has consistently been a challenge for healthcare suppliers. Cloud computing in healthcare has become the go-to choice for the management of electronic medical records.It is worthwhile for both patients as well as healthcare providers as it makes the consultation process  more consistent and spares important time. Putting away information on the cloud gives it remote accessibility and facilitates better collaboration.\par

As practices and hospitals increasingly use mobile devices to access information, from medical records and history to research and drug therapies, they are better equipped to diagnose and treat patients. New technology in healthcare plays a crucial role in preventing complications, avoiding unnecessary surgeries, improving quality of life, and sustaining health.Innovations help improve the quality and efficiency of care, develop new therapeutic approaches, treatment options, and drugs, and even predict the onset of epidemics. From ankle braces and adhesive bandages to advanced solutions such as robotic prosthetic limbs and remote heart failure monitoring and diagnostic devices, emerging technology in healthcare contributes to better patient outcomes and improved public health.\par

\end{document}